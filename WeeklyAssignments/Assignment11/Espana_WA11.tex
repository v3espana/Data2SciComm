% Options for packages loaded elsewhere
\PassOptionsToPackage{unicode}{hyperref}
\PassOptionsToPackage{hyphens}{url}
%
\documentclass[
  man,floatsintext]{apa6}
\usepackage{amsmath,amssymb}
\usepackage{iftex}
\ifPDFTeX
  \usepackage[T1]{fontenc}
  \usepackage[utf8]{inputenc}
  \usepackage{textcomp} % provide euro and other symbols
\else % if luatex or xetex
  \usepackage{unicode-math} % this also loads fontspec
  \defaultfontfeatures{Scale=MatchLowercase}
  \defaultfontfeatures[\rmfamily]{Ligatures=TeX,Scale=1}
\fi
\usepackage{lmodern}
\ifPDFTeX\else
  % xetex/luatex font selection
\fi
% Use upquote if available, for straight quotes in verbatim environments
\IfFileExists{upquote.sty}{\usepackage{upquote}}{}
\IfFileExists{microtype.sty}{% use microtype if available
  \usepackage[]{microtype}
  \UseMicrotypeSet[protrusion]{basicmath} % disable protrusion for tt fonts
}{}
\makeatletter
\@ifundefined{KOMAClassName}{% if non-KOMA class
  \IfFileExists{parskip.sty}{%
    \usepackage{parskip}
  }{% else
    \setlength{\parindent}{0pt}
    \setlength{\parskip}{6pt plus 2pt minus 1pt}}
}{% if KOMA class
  \KOMAoptions{parskip=half}}
\makeatother
\usepackage{xcolor}
\usepackage{graphicx}
\makeatletter
\def\maxwidth{\ifdim\Gin@nat@width>\linewidth\linewidth\else\Gin@nat@width\fi}
\def\maxheight{\ifdim\Gin@nat@height>\textheight\textheight\else\Gin@nat@height\fi}
\makeatother
% Scale images if necessary, so that they will not overflow the page
% margins by default, and it is still possible to overwrite the defaults
% using explicit options in \includegraphics[width, height, ...]{}
\setkeys{Gin}{width=\maxwidth,height=\maxheight,keepaspectratio}
% Set default figure placement to htbp
\makeatletter
\def\fps@figure{htbp}
\makeatother
\setlength{\emergencystretch}{3em} % prevent overfull lines
\providecommand{\tightlist}{%
  \setlength{\itemsep}{0pt}\setlength{\parskip}{0pt}}
\setcounter{secnumdepth}{-\maxdimen} % remove section numbering
% Make \paragraph and \subparagraph free-standing
\makeatletter
\ifx\paragraph\undefined\else
  \let\oldparagraph\paragraph
  \renewcommand{\paragraph}{
    \@ifstar
      \xxxParagraphStar
      \xxxParagraphNoStar
  }
  \newcommand{\xxxParagraphStar}[1]{\oldparagraph*{#1}\mbox{}}
  \newcommand{\xxxParagraphNoStar}[1]{\oldparagraph{#1}\mbox{}}
\fi
\ifx\subparagraph\undefined\else
  \let\oldsubparagraph\subparagraph
  \renewcommand{\subparagraph}{
    \@ifstar
      \xxxSubParagraphStar
      \xxxSubParagraphNoStar
  }
  \newcommand{\xxxSubParagraphStar}[1]{\oldsubparagraph*{#1}\mbox{}}
  \newcommand{\xxxSubParagraphNoStar}[1]{\oldsubparagraph{#1}\mbox{}}
\fi
\makeatother
% definitions for citeproc citations
\NewDocumentCommand\citeproctext{}{}
\NewDocumentCommand\citeproc{mm}{%
  \begingroup\def\citeproctext{#2}\cite{#1}\endgroup}
\makeatletter
 % allow citations to break across lines
 \let\@cite@ofmt\@firstofone
 % avoid brackets around text for \cite:
 \def\@biblabel#1{}
 \def\@cite#1#2{{#1\if@tempswa , #2\fi}}
\makeatother
\newlength{\cslhangindent}
\setlength{\cslhangindent}{1.5em}
\newlength{\csllabelwidth}
\setlength{\csllabelwidth}{3em}
\newenvironment{CSLReferences}[2] % #1 hanging-indent, #2 entry-spacing
 {\begin{list}{}{%
  \setlength{\itemindent}{0pt}
  \setlength{\leftmargin}{0pt}
  \setlength{\parsep}{0pt}
  % turn on hanging indent if param 1 is 1
  \ifodd #1
   \setlength{\leftmargin}{\cslhangindent}
   \setlength{\itemindent}{-1\cslhangindent}
  \fi
  % set entry spacing
  \setlength{\itemsep}{#2\baselineskip}}}
 {\end{list}}
\usepackage{calc}
\newcommand{\CSLBlock}[1]{\hfill\break\parbox[t]{\linewidth}{\strut\ignorespaces#1\strut}}
\newcommand{\CSLLeftMargin}[1]{\parbox[t]{\csllabelwidth}{\strut#1\strut}}
\newcommand{\CSLRightInline}[1]{\parbox[t]{\linewidth - \csllabelwidth}{\strut#1\strut}}
\newcommand{\CSLIndent}[1]{\hspace{\cslhangindent}#1}
\ifLuaTeX
\usepackage[bidi=basic]{babel}
\else
\usepackage[bidi=default]{babel}
\fi
\babelprovide[main,import]{english}
% get rid of language-specific shorthands (see #6817):
\let\LanguageShortHands\languageshorthands
\def\languageshorthands#1{}
% Manuscript styling
\usepackage{upgreek}
\captionsetup{font=singlespacing,justification=justified}

% Table formatting
\usepackage{longtable}
\usepackage{lscape}
% \usepackage[counterclockwise]{rotating}   % Landscape page setup for large tables
\usepackage{multirow}		% Table styling
\usepackage{tabularx}		% Control Column width
\usepackage[flushleft]{threeparttable}	% Allows for three part tables with a specified notes section
\usepackage{threeparttablex}            % Lets threeparttable work with longtable

% Create new environments so endfloat can handle them
% \newenvironment{ltable}
%   {\begin{landscape}\centering\begin{threeparttable}}
%   {\end{threeparttable}\end{landscape}}
\newenvironment{lltable}{\begin{landscape}\centering\begin{ThreePartTable}}{\end{ThreePartTable}\end{landscape}}

% Enables adjusting longtable caption width to table width
% Solution found at http://golatex.de/longtable-mit-caption-so-breit-wie-die-tabelle-t15767.html
\makeatletter
\newcommand\LastLTentrywidth{1em}
\newlength\longtablewidth
\setlength{\longtablewidth}{1in}
\newcommand{\getlongtablewidth}{\begingroup \ifcsname LT@\roman{LT@tables}\endcsname \global\longtablewidth=0pt \renewcommand{\LT@entry}[2]{\global\advance\longtablewidth by ##2\relax\gdef\LastLTentrywidth{##2}}\@nameuse{LT@\roman{LT@tables}} \fi \endgroup}

% \setlength{\parindent}{0.5in}
% \setlength{\parskip}{0pt plus 0pt minus 0pt}

% Overwrite redefinition of paragraph and subparagraph by the default LaTeX template
% See https://github.com/crsh/papaja/issues/292
\makeatletter
\renewcommand{\paragraph}{\@startsection{paragraph}{4}{\parindent}%
  {0\baselineskip \@plus 0.2ex \@minus 0.2ex}%
  {-1em}%
  {\normalfont\normalsize\bfseries\itshape\typesectitle}}

\renewcommand{\subparagraph}[1]{\@startsection{subparagraph}{5}{1em}%
  {0\baselineskip \@plus 0.2ex \@minus 0.2ex}%
  {-\z@\relax}%
  {\normalfont\normalsize\itshape\hspace{\parindent}{#1}\textit{\addperi}}{\relax}}
\makeatother

\makeatletter
\usepackage{etoolbox}
\patchcmd{\maketitle}
  {\section{\normalfont\normalsize\abstractname}}
  {\section*{\normalfont\normalsize\abstractname}}
  {}{\typeout{Failed to patch abstract.}}
\patchcmd{\maketitle}
  {\section{\protect\normalfont{\@title}}}
  {\section*{\protect\normalfont{\@title}}}
  {}{\typeout{Failed to patch title.}}
\makeatother

\usepackage{xpatch}
\makeatletter
\xapptocmd\appendix
  {\xapptocmd\section
    {\addcontentsline{toc}{section}{\appendixname\ifoneappendix\else~\theappendix\fi: #1}}
    {}{\InnerPatchFailed}%
  }
{}{\PatchFailed}
\makeatother
\keywords{keywords\newline\indent Word count: X}
\usepackage{lineno}

\linenumbers
\usepackage{csquotes}
\ifLuaTeX
  \usepackage{selnolig}  % disable illegal ligatures
\fi
\usepackage{bookmark}
\IfFileExists{xurl.sty}{\usepackage{xurl}}{} % add URL line breaks if available
\urlstyle{same}
\hypersetup{
  pdftitle={Proportional Reasoning Across Formats},
  pdfauthor={Vincent Espana1},
  pdflang={en-EN},
  pdfkeywords={keywords},
  hidelinks,
  pdfcreator={LaTeX via pandoc}}

\title{Proportional Reasoning Across Formats}
\author{Vincent Espana\textsuperscript{1}}
\date{}


\shorttitle{Proportional Reasoning}

\authornote{

The authors made the following contributions. Vincent Espana: Conceptualization, Writing - Original Draft Preparation, Writing - Review \& Editing.

Correspondence concerning this article should be addressed to Vincent Espana, 123 Rutgers Way, Piscataway, NJ 08854. E-mail: \href{mailto:vincent.espana@rutgers.edu}{\nolinkurl{vincent.espana@rutgers.edu}}

}

\affiliation{\vspace{0.5cm}\textsuperscript{1} Rutgers University}

\begin{document}
\maketitle

\section{Introduction}\label{introduction}

Comparing proportions is sometimes very hard! But, even infants seem to be able to do it a little bit. The purpose of this science project was to better understand how well people compare proportions when the proportions are presented in different formats. By analyzing performance across different factors, we hope to identify if said factors affect proportional reasoning. Specifically, this study focuses on three research objectives. First, we examine whether average performance varies depending on format type. Second, we investigate whether average performance varies across numerator congruency status. Lastly, we explore the interaction between numerator congruency and format type to determine if congruency is affected by differences in formats.

The purpose of this class assignment is to take the R-code and plots we've been generating over the last several weeks and put it all together into one APA-formatted paper.

\section{Methods}\label{methods}

A total of 99 adults participated in the study. Participants were introduced to a story about how a magic ball and that the outcome (i.e.,blue or orange) depended on the proportions. They were then asked to compare the proportions of different images. In other words, participants were shown two images of the same kind at the same time and asked to decide which image had a higher proportion of the shape (or dots) colored in blue. See Figure \ref{fig:trial}.

\begin{figure}
\centering
\includegraphics{Probtask_Trial.png}
\caption{\label{fig:trial}Example of Proportional Reasoning Task}
\end{figure}

\subsection{Conditions}\label{conditions}

There were four different conditions that changed what kinds of images the participants saw. (See Figure \ref{fig:formats})

\begin{itemize}
\tightlist
\item
  divided blob: blue and orange were entirely separate
\item
  integrated blob: one blob, divided to be part blue and part orange
\item
  separated dots: blue and orange dots were on opposite sides of the image
\item
  integrated dots: blue and orange dots were intermixed
\end{itemize}

\begin{figure}
\centering
\includegraphics{Probtask_formats.png}
\caption{\label{fig:formats}Example Images Per Condition}
\end{figure}

\subsection{Data analysis}\label{data-analysis}

We used R (Version 4.4.1; R Core Team, 2024) and the R-packages \emph{dplyr} (Version 1.1.4; Wickham, Francois, Henry, Muller, \& Vaughan, 2023), \emph{forcats} (Version 1.0.0; Wickham, 2023a), \emph{ggplot2} (Version 3.5.1; Wickham, 2016), \emph{lubridate} (Version 1.9.3; Grolemund \& Wickham, 2011), \emph{papaja} (Version 0.1.3; Aust \& Barth, 2024), \emph{purrr} (Version 1.0.2; Wickham \& Henry, 2023), \emph{readr} (Version 2.1.5; Wickham, Hester, \& Bryan, 2024), \emph{stringr} (Version 1.5.1; Wickham, 2023b), \emph{tibble} (Version 3.2.1; Müller \& Wickham, 2023), \emph{tidyr} (Version 1.3.1; Wickham, Vaughan, \& Girlich, 2024), \emph{tidyverse} (Version 2.0.0; Wickham et al., 2019) and \emph{tinylabels} (Version 0.2.4; Barth, 2023) for all our analyses.

\section{Results}\label{results}

\begin{enumerate}
\def\labelenumi{\arabic{enumi}.}
\tightlist
\item
  Does average performance vary across format type, ignoring all other aspects of the stimuli?
\end{enumerate}

\begin{figure}
\centering
\includegraphics{Espana_WA11_files/figure-latex/plot-one-1.pdf}
\caption{\label{fig:plot-one}Mean Accuracy and Accuracy Variability across Conditions}
\end{figure}

Figure \ref{fig:plot-one} compares how accurately participants performed under each condition. It shows that stacked blob, and random dots have a even distribution, while shifted blob clusters to at higher and lower accuracy and separated dots cluster at lower accuracy. The means of these condition show that no condition is more significant than another as the means are relatively similar.

\begin{enumerate}
\def\labelenumi{\arabic{enumi}.}
\setcounter{enumi}{1}
\tightlist
\item
  How are reaction time and accuracy related?
\end{enumerate}

\begin{figure}
\centering
\includegraphics{Espana_WA11_files/figure-latex/plot-two-1.pdf}
\caption{\label{fig:plot-two}Scatter Plot of Reaction Time vs.~Proportion Correct by Condition}
\end{figure}

As seen in Figure \ref{fig:plot-two}, there is a positive relationship between the average reaction time and the proportion correct. When looking at the conditions individually, shifted blob and separated dots cluster toward a lower reaction time and a lower accuracy and stacked blob and random dots cluster toward a higher accuracy and higher reaction time.

\begin{enumerate}
\def\labelenumi{\arabic{enumi}.}
\setcounter{enumi}{2}
\tightlist
\item
  How does numerator congruency interact with format type?
\end{enumerate}

\begin{figure}
\centering
\includegraphics{Espana_WA11_files/figure-latex/plot-three-1.pdf}
\caption{\label{fig:plot-three}Mean Proportion Correct and Variability across Condition and Numerator Congruency}
\end{figure}

Figure \ref{fig:plot-three} compares the proportion correct per condition of congruent numerators and incongruent numerators. The graph shows that congruent proportions cluster towards a higher accuracy while the incongruent proportions have a high variability and a low mean accuracy.

\section{Discussion}\label{discussion}

\subsection{Interpretation}\label{interpretation}

\begin{enumerate}
\def\labelenumi{\arabic{enumi}.}
\tightlist
\item
  Does average performance vary across format type?
\end{enumerate}

Yes, average performance varies across format type, with stacked blob and random dots having a higher overall mean. However, as seen in Figure \ref{fig:plot-one}, there is not a significant difference between them.

\begin{enumerate}
\def\labelenumi{\arabic{enumi}.}
\setcounter{enumi}{1}
\tightlist
\item
  Does average performance vary across numerator congruency status?
\end{enumerate}

As seen in Figure \ref{fig:plot-three}, congruent proportions have a higher overall mean and better variability than incongruent proportions, thus alluding to a significant distinction.

\begin{enumerate}
\def\labelenumi{\arabic{enumi}.}
\setcounter{enumi}{2}
\tightlist
\item
  Does numerator congruency vary across format type? (i.e., is there an interaction)
\end{enumerate}

Figure \ref{fig:plot-three} also suggests that there is an interaction, as the numerator congruency has variability across the different type. More specifically, congruency has the same trend across each condition, but the conditions seem to affect the mean of the incongruent proportions the most.

\subsection{Conclusion}\label{conclusion}

\begin{enumerate}
\def\labelenumi{\arabic{enumi}.}
\tightlist
\item
  What was the most annoying or hardest thing about the assignment?
\end{enumerate}

The most annoying part of the assignment was resizing Figure 2 to fit the poster dimensions. I was using a different way to import it initially, but once I found a better syntax, I was able to finally resize to fit the paper.

\begin{enumerate}
\def\labelenumi{\arabic{enumi}.}
\setcounter{enumi}{1}
\tightlist
\item
  What was the most satisfying or fun thing about the assignment?
\end{enumerate}

The most satisfying part of this assignment is seeing the final result as one cohesive poster. Even though I did not specifically format each part of it, it is interesting to see how our work can be implemented in a much better way than reformatting a Word document or Google doc. The possibilities for creating seem endless!

\newpage

\section{References}\label{references}

\phantomsection\label{refs}
\begin{CSLReferences}{1}{0}
\bibitem[\citeproctext]{ref-R-papaja}
Aust, F., \& Barth, M. (2024). \emph{{papaja}: {Prepare} reproducible {APA} journal articles with {R Markdown}}. \url{https://doi.org/10.32614/CRAN.package.papaja}

\bibitem[\citeproctext]{ref-R-tinylabels}
Barth, M. (2023). \emph{{tinylabels}: Lightweight variable labels}. Retrieved from \url{https://cran.r-project.org/package=tinylabels}

\bibitem[\citeproctext]{ref-R-lubridate}
Grolemund, G., \& Wickham, H. (2011). Dates and times made easy with {lubridate}. \emph{Journal of Statistical Software}, \emph{40}(3), 1--25. Retrieved from \url{https://www.jstatsoft.org/v40/i03/}

\bibitem[\citeproctext]{ref-R-tibble}
Müller, K., \& Wickham, H. (2023). \emph{Tibble: Simple data frames}. Retrieved from \url{https://CRAN.R-project.org/package=tibble}

\bibitem[\citeproctext]{ref-R-base}
R Core Team. (2024). \emph{R: A language and environment for statistical computing}. Vienna, Austria: R Foundation for Statistical Computing. Retrieved from \url{https://www.R-project.org/}

\bibitem[\citeproctext]{ref-R-ggplot2}
Wickham, H. (2016). \emph{ggplot2: Elegant graphics for data analysis}. Springer-Verlag New York. Retrieved from \url{https://ggplot2.tidyverse.org}

\bibitem[\citeproctext]{ref-R-forcats}
Wickham, H. (2023a). \emph{Forcats: Tools for working with categorical variables (factors)}. Retrieved from \url{https://CRAN.R-project.org/package=forcats}

\bibitem[\citeproctext]{ref-R-stringr}
Wickham, H. (2023b). \emph{Stringr: Simple, consistent wrappers for common string operations}. Retrieved from \url{https://CRAN.R-project.org/package=stringr}

\bibitem[\citeproctext]{ref-R-tidyverse}
Wickham, H., Averick, M., Bryan, J., Chang, W., McGowan, L. D., Françoi, R., \ldots{} Hiroaki Yutani. (2019). Welcome to the {tidyverse}. \emph{Journal of Open Source Software}, \emph{4}(43), 1686. \url{https://doi.org/10.21105/joss.01686}

\bibitem[\citeproctext]{ref-R-dplyr}
Wickham, H., Francois, R., Henry, L., Muller, K., \& Vaughan, D. (2023). \emph{Dplyr: A grammar of data manipulation}. Retrieved from \url{https://CRAN.R-project.org/package=dplyr}

\bibitem[\citeproctext]{ref-R-purrr}
Wickham, H., \& Henry, L. (2023). \emph{Purrr: Functional programming tools}. Retrieved from \url{https://CRAN.R-project.org/package=purrr}

\bibitem[\citeproctext]{ref-R-readr}
Wickham, H., Hester, J., \& Bryan, J. (2024). \emph{Readr: Read rectangular text data}. Retrieved from \url{https://CRAN.R-project.org/package=readr}

\bibitem[\citeproctext]{ref-R-tidyr}
Wickham, H., Vaughan, D., \& Girlich, M. (2024). \emph{Tidyr: Tidy messy data}. Retrieved from \url{https://CRAN.R-project.org/package=tidyr}

\end{CSLReferences}


\end{document}
